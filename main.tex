% !TeX program = xelatex
\documentclass[12pt,a4paper]{article}

\usepackage{polyglossia}
\setdefaultlanguage{english}
\setotherlanguages{hindi,sanskrit,bengali}

\usepackage{fontspec}
\setmainfont{Times New Roman}
\newfontfamily{\devanagarifont}[Script=Devanagari]{Lohit Devanagari.ttf}
\newfontfamily{\bengalifont}[Script=Bengali]{kalpurush.ttf}

\title{\textbengali{মাজার-মানত-পূজন, সনাতনী ভাবনায়}}
\author{\textbengali{বিপুল মোহন্ত }}
\date{}
\begin{document}\maketitle\vskip-1.0cm\hrulefill
	
	
\section*{\textbengali{সারকথা}} 
\textbengali{২০১১ সালে ছোট করে মাজার সম্পর্কে চিরায়িত হিন্দুদের ভয়-ভক্তি সম্পর্কে একখানা প্রবন্ধ রচনা করেছিলাম, ২০০১ সালে দেখা আমার ব্যক্তিগত অভিজ্ঞতার আলোকে। প্রবন্ধটি অনেক জ্ঞানী হিন্দুগণের দৃষ্টি আকর্ষণ করতে সক্ষম হয়েছিল, অনেকে ব্যক্তিগত ভাবে আমাকে উৎসাহিত করেছিলেন এই ধরণের আরও লেখার জন্য। আজকের এই প্রবন্ধটি সেদিনের সেই প্রবন্ধের ২য় সংস্করণ, ঐতিহাসিক, সামাজিক ও ধর্মীয় দায়বদ্ধতার ভিত্তিতে। উল্লেখ্য, আমার বর্তমান লেখনী আমি বাংলাকেন্দ্রিক রাখছি, ভবিষ্যতে আরও ব্যাপকতা নিয়ে আলোচনা হবে।   আশা করি প্রবন্ধ পাঠের পর আপনি আপনার নিজস্বতা দিয়ে বিবেচনা করে নির্দিষ্ট উপসঙ্ঘারে উপনিত হবেন।   } 

\section*{\textbengali{সূচনায় খলজিনামা}}
\textbengali{বাংলায় হিন্দু শাসনের পতন এবং মুসলিম শাসনের সূচনা ঘটেছিল ইখতিয়ার উদ্দিন মুহাম্মাদ বখতিয়ার খলজি নামক একজন আফগান সেনাপতির হাত ধরে। সে বাংলা এবং বিহার জয় করেছিল তৎকালীন হিন্দু রাজাদের কাপুরুষতা এবং পারস্পরিক বিভেদকে কাজে লাগিয়ে, সাল ১২০৪। তার অভিযানর লক্ষ্য যে শুধুমাত্র রাজ্য বিজয় ছিলনা সে বিষয়ে পণ্ডিতগণ একমত। ইতিহাস থেকে জানা যায়, পূর্ব ভারতে তার অভিযানের সময়ই আলিমদের ইসলামী দাওয়াতের কাজ সর্বাধিক সাফল্য অর্জন হয়েছিলো এবং ভারতীয় উপমহাদেশের ইতিহাসে বাংলায় সবচেয়ে বেশি মানুষ ইসলাম ধর্ম গ্রহণ করেছিল। খলজির পূর্বপুরুষেরা ছিল তুর্কীস্তানের অধিবাসী। তথাকথিত জিহাদি এই যোদ্ধা ১১৯৩ সালে নালন্দা বিশ্ববিদ্যালয় পুরোপুরি ধ্বংস করে। তৎকালীন বাংলার রাজা লক্ষণ সেনকে অতর্কিত আক্রমনের মাধ্যমে বাংলা জয় করে।
	
	
	 
}

\end{document}